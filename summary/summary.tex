\documentclass[a4paper]{scrreprt}

% Uncomment to optimize for double-sided printing.
% \KOMAoptions{twoside}

% Set binding correction manually, if known.
% \KOMAoptions{BCOR=2cm}

% Localization options
\usepackage[english]{babel}
\usepackage[T1]{fontenc}
\usepackage[utf8]{inputenc}

% Sub figures
\usepackage{subcaption}

% Quotations
\usepackage{dirtytalk}

% Floats
\usepackage{float}

% Enhanced verbatim sections. We're mainly interested in
% \verbatiminput though.
\usepackage{verbatim}

% Automatically remove leading whitespace in lstlisting
\usepackage{lstautogobble}

% CSV to tables
\usepackage{csvsimple}

% PDF-compatible landscape mode.
% Makes PDF viewers show the page rotated by 90°.
\usepackage{pdflscape}

% Advanced tables
\usepackage{array}
\usepackage{tabularx}
\usepackage{longtable}

% Fancy tablerules
\usepackage{booktabs}

% Graphics
\usepackage{graphicx}

% Current time
\usepackage[useregional=numeric]{datetime2}

% Float barriers.
% Automatically add a FloatBarrier to each \section
\usepackage[section]{placeins}

% Custom header and footer
\usepackage{fancyhdr}

\usepackage{geometry}
\usepackage{layout}

% Math tools
\usepackage{mathtools}
% Math symbols
\usepackage{amsmath,amsfonts,amssymb}
\usepackage{amsthm}
% General symbols
\usepackage{stmaryrd}

% Utilities for quotations
\usepackage{csquotes}

% Bibliography
\usepackage[
  style=alphabetic,
  backend=biber, % Default backend, just listed for completness
  sorting=ynt % Sort by year, name, title
]{biblatex}
\addbibresource{references.bib}

\DeclarePairedDelimiter\abs{\lvert}{\rvert}
\DeclarePairedDelimiter\floor{\lfloor}{\rfloor}

\pagestyle{plain}

% Source code & highlighting
\usepackage{listings}

% SI units
\usepackage[binary-units=true]{siunitx}
\DeclareSIUnit\cycles{cycles}

% Should use this command wherever the print date is mentioned.
\newcommand{\printdate}{\today}

\newcommand{\mailsubject}{63112 Data Management Data Structures - Summary}
\newcommand{\maillink}[1]{\href{mailto:#1?subject=\mailsubject}
                               {#1}}

\subject{63112 Data Management Data Structures}
\title{Summary}

\author{Michael Senn \maillink{michael.senn@students.unibe.ch} --- 16-126-880}

\date{\printdate}

% Needs to be the last command in the preamble, for one reason or
% another. 
\usepackage{hyperref}

\begin{document}
\maketitle

\chapter{Introduction}

\begin{itemize}
		\item Different ways to classify data structures
				\begin{itemize}
						\item Probabilistic structures either trade off some
								incorrect queries for a more compact
								representation (e.g. bloom filter), or use
								randomness for data layout (e.g. skiplist)
						\item Persistent structures have built-in capabitlities
								to offload parts, or all of it, to storage
						\item Non-persistent ones assume they will live fully
								in memory
						\item Scalable
						\item Compressed
						\item Hash-based, Graph-Based, ...
				\end{itemize}
		\item Generic data structures able to store anything, as well as
				specific ones for e.g. text, grpah, geo data, ...
\end{itemize}

\chapter{B trees}

\section{Motivation}

\begin{itemize}
		\item Fast lookups over data which might not fit in memory
		\item Efficient insertion and deletion (compared to e.g. sorted array)
\end{itemize}

\section{Variations}

\begin{itemize}
		\item B* tree: Each node $\geq 2/3$ full
		\item B+ tree: Data only in leaf nodes
		\item B-link tree: Nodes on each level form linked list
		\item Often, tree with all these features also called B+ tree
\end{itemize}

\section{Layout}

\begin{itemize}
		\item Leaf nodes store data, either directly, or as pointer to actual
				resource
		\item Intermediary nodes contain separator keys which split data range
				of leaf nodes down their respective paths
		\item Large fanout factor (e.g. 100) means most space used for leaf
				nodes. Allows to e.g. cache all non-leaf nodes in memory even
				for gigantic tree.
\end{itemize}

\section{Operations}

\begin{itemize}
		\item Lookup traverses tree using separator keys until leaf node found
		\item Range lookups can then use leaf-level links to travel to next leaf
		\item On insertion, node might have to be split in two, with new separator key
		\item On deletion, nodes might have to be merged. Or we simply don't do this
\end{itemize}

\section{Buffer pool}

\begin{itemize}
		\item Allows retrieving pages by logical ID
		\item Allows retrieving fresh page
		\item Can be asked to pin/unpin/flush pages
		\item Handles associated disk IO if page not loaded
		\item Caches pages (LRU, random, ...) in memory
\end{itemize}

\subsection{Optimizations}

\begin{itemize}
		\item Pointer swizzling: Pointers to page IDs can, when page is loaded
				(and they point to a page which is /also/ loaded), be turned
				into direct pointers to memory
\end{itemize}

\section{Various}

\begin{itemize}
		\item Read (write) amplification: Actual amount of data read (written)
				compared to data we meant to read (write)
		\item B+-tree is read-oriented
		\item Unsuitable for variable keys (no stable fence keys) or large keys
				(not many keys per page)
\end{itemize}

\chapter{LSM trees}

\section{Motivation}

\begin{itemize}
		\item Write-oriented structure which still must support somewhat-efficient reads
		\item E.g. logging sensor metrics, or transaction logs
		\item In a B-tree a single 16B insert might cause 2-3 IOPS and 4KB of
				data written. Not ideal.
\end{itemize}

\section{Structure}

\textbf{Key insight} defer putting new data to persistent storage. Two
components: One memory-resident, one disk-resident.

\begin{itemize}
		\item $C_0$ tree: In-memory component (very freeing, needn't care about
				pages). Commonly done with e.g. Skiplist or AVL tree.
		\item Inserts done into $C_0$ tree (and logged, for resiliency)
		\item $C_0$ periodically compacted (merged) with on-disk $C_1$ component
		\item With inserts meanwhile going into new $C_0$ component
		\item $C_1$ can be done as fully loaded B-Tree. Or SSTables.
		\item SSTables: Shallow tree for variable-length keys.
\end{itemize}

\subsection{SSTable layout (example)}

\begin{itemize}
		\item Multi-level structure.
		\item Metadata of each SSTable: Key range (min, max), bloom filter,
				fixed-size index mapping key to approximate location in file.
		\item Fixed-size index can be achieved by e.g. using potentially
				shortened separator keys, or subset of actual keys.
		\item Lookup thus requires some traversal in structure
		\item Actual data is sorted values, e.g. key-value pairs sorted by key.
\end{itemize}

\section{Operations}

\begin{itemize}
		\item Insertion: Log (resiliency), the insert to $C_0$
		\item Compaction: Once $C_0$ full, flush to first disk tier. Write
				*new* SSTable.
		\item Merging: When $C_k$ full, merges entries of $C_k$ and produces
				non-overlapping SSTables for $C_{k+1}$ tier
		\item Lookup: Starting top down, for each SSTable: First check range.
				Then bloom filter. Then approximate location using fixed-size
				index. Then scan until data (or tombstone) found. Proceed to
				next level if not found.
		\item Deletion using tombstones
\end{itemize}

\chapter{Probabilistic structures}

\section{Motivation}

\begin{itemize}
		\item Use non-determinism for either a more compact representation (at
				the cost of inaccurate queries)
		\item Or for determining the topology of the data structure
		\item Index lookups can be expensive if they don't produce a hit. Goal:
				Structure in front which can cheaply catch large amount of
				these before they hit the index.
\end{itemize}

\section{Excerpt Hashing}

\begin{itemize}
		\item Numerical hashing (division): $h(x) = x \bmod m$, choice of $m$
				essential.  Primes not too close to power of two good.
		\item Numerical hashing (multiplication): $h(x) = \floor{m \cdot A}
				\bmod m$, choice of $m$ less important. $A$ can be e.g. golden
				ratio. But more expensive than division.
		\item String hashing: Use characters as coefficients of polynomial
				(rolling around).
		\item String hashing: Or FNV, MurmurHash3, very-fast hash functions
		\item All of these non-cryptographic, e.g. no collision resistance!
\end{itemize}

\section{Bloom filters}

\begin{itemize}
		\item Bitmap with $m$ bits
		\item $k$ hash functions which map to one of $m$ bits (just take
				regular hash functions $\bmod m$)
		\item To add element, hash with each function. Set bits which it maps to.
		\item To check, equivalently
		\item Can make hash functions use disjoint sets of bitmaps to prevent
				contention during bit access.
		\item More space for bitmap lowers false positiity rate. Or, keeping
				false positivity rate equal, allows using fewer hash functions
				(= faster).
		\item Issues: No resizing, no deletion, no data locality so must be
				in-memory for reasonable performance.
\end{itemize}

\section{Counting data structures}

\begin{itemize}
		\item Motivation: Keep track of unique visitors, or cardinality of a column
		\item Naive approach: Keep list of sorted values. But then $O(log n)$
				insertion and $O(n)$ space.
		\item Alternatives: Hashing will bring updates down to constant time,
				but still linear space. Sampling will trade off accuracy
				(especially for low-frequency values!) for lower space usage.
\end{itemize}

\subsection{Linear counting}

\begin{itemize}
		\item Bitmap with $m$ bits
		\item To insert, hash with one function and set corresponding bit
		\item With $z$ bits set, estimator of count is then $-m \cdot ln(1 - z/m)$
		\item Accuracy depends on load factor, so cardinality must be roughly known upfront
		\item Required space to keep error low is proportional to cardinality
\end{itemize}

\subsection{Probabilistic counting}

\begin{itemize}
		\item Instead of setting bit $h(x)$, set the bit corresponding to the rightmost $1$ in $h(x)$
		\item Then bits on left are way less likely to be set than bits on right (same idea as hash-based POW)
		\item Given leftmost $0$ at position $n$, estimator is then $2^n/0.77$
		\item But expected margin of error is 78 percent
		\item Can be offset by using several such counters. Use hash output to select which counter to update.
		\item Average of these counters can lower error margin to 10 percent for 64 maps.
\end{itemize}



\printbibliography

\end{document}
